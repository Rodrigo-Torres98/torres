\newpage
\begin{center}
  РЕФЕРАТ
\end{center}

Объем работы 55 стр. Работа содержит 9 иллюстрации, 4 таблицы, 10 библиографических источников и 44 графические листы.
материалы Количество запросов: 2. Графический материал предоставлен
см. приложение А. Схема узла, включая подключение компонентов, представлена ​​в приложении Б.
Список ключевых слов: веб-страница, клиническая лаборатория, пользователь, врач, пациент, специальность, регистрация, пароль, дизайн, облегчение, база данных, ветеринария, животные, программное обеспечение, список.
Целью разработки является веб-сайт больничной системы для клинического приема животных для осмотра и получения информации о животных с квалифицированными ветеринарными врачами и
Запись выполненных процедур и диагнозов для обновления информации.
В процессе создания сайта были определены основные объекты путем создания информационных блоков, использованы классы и модули с функциями, обеспечивающими работу с объектами предметной области, а также корректная работа сайта. разработаны разделы, содержащие информацию о ветеринарах, владельцах животных и особенностях животных.
Целью дипломной квалификационной работы является оцифровка информации, заданий, отказ от использования листов бумаги и гибкость, оперативность и результативность для ввода соответствующих данных о животных.
Сайт разработан на языке программирования C\# с применением HTML и CSS реализации.


\newpage
\selectlanguage{english}
\begin{center}ABSTRACT\end{center}
  

The volume of work is 55 pages. The work contains 9 illustrations, 4 tables, 10 bibliographical sources and 44 graphic sheets.
materials Number of requests: 2. Graphic material provided
see appendix A. The layout of the site, including the connection of components, is presented in appendix B.
Keywords list: web page, clinical laboratory, user, doctor, patient, specialty, registration, password, design, facilitate, database, veterinary, animals, software, list.
The object of the development is the website of a hospital system to clinically admit animals to examine and consult the information of the animals, with qualified veterinary doctors and
Record of procedures performed and diagnoses, to update the information.
In the process of creating the site, the main objects were identified by the creation of information blocks, classes and modules were used with functions that provide work with objects of the subject area, and as well as the correct operation of the website, have been developed sections, containing information on veterinarians, animal owners, and animal specialties.
The purpose of the thesis qualification work is the digitalization of information, tasks, eliminating the use of sheets of paper and the flexibility, efficiency and effectiveness to enter relevant data of the animals.
The website was developed using the C\# programming language with the use of HTML and CSS implementation.
\selectlanguage{russian}
