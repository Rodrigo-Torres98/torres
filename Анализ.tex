\newsection
\section{Анализ предметной области}
\subsection{Характеристики компании и пользователей}

Хотя первый веб-сервер был запущен в августе 1991 года, правда в том, что Интернет родился в 70-х годах. Имейте в виду, что в то время передача данных была не слишком быстрой, но достаточно эффективной. Это привело к тому, что его использовали врачи, учителя, ученые, правительственные чиновники и даже военные. Таким образом, информация могла передаваться внутри определенной группы.

В этот новый сезон было необходимо иметь компьютер, модем и телефонную линию. С наступлением 90-х годов появилась Mosaic, бесплатная программа, которая стала первым коммерческим браузером, с помощью которого общественность могла получить доступ к онлайн-контенту. Он был разработан Эриком Биной и Марком Андизеном и первоначально работал на sistena Unix. Уже в 1994 году он стал доступен для большего количества операционных систем, таких как Windows или Mac. Хочешь узнать больше?

В 90-е годы, в дополнение к тому, что все подключались к Интернету по телефонным линиям, файлы просматривались на языке html. Таким образом, можно было легко получить доступ к файлам и документам в сети. Ранние сайты отличались ограниченным количеством цветов, огромным количеством текста и линейно размещенной графикой. Эти сайты известны как"сайты первого поколения". Эти версии html были предназначены для функционального общения, а не для развлечения, в отличие от того, что происходит сегодня. С другой стороны, было бы невозможно загрузить графику высокой четкости и, конечно же, видео. Вы можете себе представить причину? Скорость соединения все еще была очень низкой.

\subsection{Html, первое поколение в мире веб-дизайна}
В 60-х годах Тед Нельсон был ответственен за создание концепции гипертекста, но только несколько десятилетий спустя она начала использоваться. Начиная с 80-х годов Тим Бернерс Ли предложил проект, основанный на html, создав в 1991 году первую веб-страницу и первый браузер, который он назвал httpd. Там вы могли найти подробную информацию о создании сайта и советы по поиску информации на нем, поскольку поисковых систем еще не существовало. 

В 1994 году родилась Всемирная паутина или www, и html был единственным способом кодирования для веб-дизайна. Но правда в том, что на момент их разработки не было слишком большой свободы. Из-за этого можно было создавать только очень простые макеты с таблицами, текстами и некоторыми ссылками. Но уже в то время дизайнеры думали о том, как сделать сайты намного более привлекательными.

Чтобы помешать самым влиятельным компаниям монополизировать весь бизнес веб-дизайна, был создан W3C. И дело в том, что если бы существовала только одна компания, контролирующая кодирование, история веб-дизайна была бы совсем другой. Даже сегодня W3C отвечает за организацию стандартов, руководящих принципов и кодификации для создания веб-сайта высочайшего качества. Важно соблюдать каждое из предписанных вами правил, чтобы оптимизировать сайты с точки зрения качества и чистоты программирования. Это означает безупречную работу. После Mosaic в ноябре 1994 года появился Netscape, который начал создавать свои собственные лейблы, а www последовал за ним.

\subsection{Второе поколение}
В период с 1992 по 1994 год появился html 2 с гораздо более быстрым подключением и веб-дизайнерами с большим количеством опций html и большими возможностями. Код может похвастаться большим количеством графики и приобретает все большую сложность. Важный шаг вперед происходит, когда некоторые значки начинают использоваться вместо слов, а сайты начинают предлагать фоновые изображения. Также появляется использование баннеров и кнопок, а также возможность упорядочивать тексты с помощью списков и меню. Видеокарты улучшают разрешение и цветопередачу, что обеспечивает высокое качество дизайна. Ты хочешь продолжать двигаться вперед?

\subsection{Третье поколение}
В 1995 году появилось больше html-тегов, а вместе с ними и html 3. Теперь у вас появилось больше возможностей при разработке веб-сайтов, таких как использование таблиц и таблиц со стилями CSS. В том же году появился скандальный и популярный браузер Microsoft: Internet Explorer. В это время вводится больше возможностей для улучшения эстетической и визуальной части дизайна. Эти сайты третьего поколения уже могут иметь цветной фон, а анимация представлена с появлением нового формата, известного как gif. Учтите, что именно на этом этапе работа веб-дизайнеров начинает выделяться. Страницы уже разработаны в соответствии с потребностями каждого из них. Дизайн начинает адаптироваться к своим функциям, и появляются первые рекламные страницы.

Также в 1995 году появился JavaScript для устранения определенных ограничений языка html. С его появлением дизайн приобрел большую динамичность, но, тем не менее, загрузка страниц стала намного медленнее. В настоящее время многие эксперты избегают его использования и предпочитают использовать CSS, но правда в том, что он по-прежнему прочно поддерживается как во внешнем, так и во внутреннем интерфейсе. Первый - это все, что связано с тем, что отображается на веб-сайте. Серверная часть, со своей стороны, гарантирует отправку правильных данных в любой браузер, дополнительно создавая функциональность системы.

В 1996 году появилась Flash, а вместе с ней и свобода создавать более впечатляющие дизайны, с возможностью добавлять визуальные эффекты и преодолевать барьеры, которые до этого существовали в мире дизайна страниц. Это была мечта, осуществленная благодаря единственному инструменту. Этой эпохе удалось стать важной вехой, но со временем стало ясно, что взаимодействия и эффекты были недоступны для поиска, который индексировался с помощью html; кроме того, он потреблял вычислительную мощность, по причинам, по которым со временем он потерял силу.

Мы рассказываем вам, что именно в 1998 году начался бум CSS. Этот язык стал очень популярным благодаря своему предложению, которое отделяет содержание от форм презентаций. Таким образом, в html обрабатывается содержимое, а в CSS определяется внешний вид и форматирование. Процесс его развития был отложен на несколько лет, пока не были достигнуты наилучшие результаты. И дело в том, что изначально визуализация варьировалась в зависимости от используемого браузера, что затрудняло работу разработчиков.

\subsection{Четвертое поколение веб-сайтов}
Современные сайты известны как сайты четвертого поколения. В дополнение к языку html, который является основной частью всей структуры, дизайнеры также имеют широкий спектр возможностей и языков на выбор, таких как Javascript, ASP, Flash, XML и CSS и другие. Благодаря этим технологиям появились социальные сети, форумы, блоги и чаты, содержащие видео и аудио.

Фактически, в 2003 году начинается эра информации, ориентированной на любого пользователя. Веб 2.0, наряду с блогами и социальными сетями, становится очень популярным. Его интерфейсы намного приятнее визуально, и он вступает в более развитую фазу, когда дело доходит до веб-дизайна. Это момент, когда вы начинаете думать как о поисковых системах, так и о пользователях.

\subsection{Пятое поколение}
Мы могли бы добавить пятое поколение, которое отражало бы популярность телевизионных онлайн-страниц. Также наблюдается рост облачных вычислений, которые представляют операционные системы, приложения, такие как текстовые редакторы и т. Д., В облаке. Обратите внимание: html 5-это пересмотр этого языка, который используется сегодня. Вводятся новые теги, улучшающие семантику документа, такие как article, header или section. Благодаря этим улучшениям некоторые дополнительные плагины, такие как Java и Flash, становятся менее необходимыми. Это решение большинства проблем и синтаксических ошибок, с которыми мы сталкивались до сих пор. Кроме того, он совместим с различными браузерами, такими как Apple Safari, Mozilla Firefox, Google Chrome или Internet Explorer.

\subsection{Аддитивные технологии, их классификация}

Основное преимущество веб-разработки и sofwore заключается в том, что на сегодняшний день существует множество инструментов, которые помогают программисту В результате, нет необходимости планировать последовательность технологических процессов, специальное оборудование для обработки материалов, транспортировка от машины к машине и т. Д.

Разработка включает в себя такие методы, как многоуровневое слияние и многоплатформенная разработка.
Android. Разработка, которая сейчас пользуется большим спросом, потому что создание для мобильных устройств было очень полезным.

\subsection{Преимущества использования HTML}
HTML — это стандартный язык разметки для создания веб-страниц и приложений. С HTML вы можете создать свою собственную веб-страницу.

\begin{itemize}
\item Это легко узнать: вам понравится!
\item Это язык разметки, что означает, что он используется для описания структуры веб-страниц. HTML не является языком программирования, а это означает, что его нельзя использовать для создания динамического веб-контента.
\item HTML — это статический язык, а это означает, что веб-страницы, созданные с помощью HTML, нельзя изменить без ручного редактирования кода HTML.
\item Полезно для создания статических веб-страниц, которые подходят для небольших веб-сайтов или не требуют частых обновлений.
\end{itemize}

HTML — это язык гипертекстовой разметки, используемый для создания веб-страниц. Он имеет много преимуществ, таких как возможность создавать интерактивные и мультимедийные веб-страницы.

Однако он также имеет некоторые недостатки, такие как необходимость использования совместимого браузера для просмотра веб-страниц, созданных с помощью HTML. В целом, HTML — это мощный и универсальный язык разметки, который можно использовать для создания привлекательных и интересных веб-страниц.

Основная панель навигации шаблона содержит:
\begin{itemize}
\item  Пациенты;
\item  Животные;
\item  отчеты;
\item  Врачи.
\end{itemize}
Программное обеспечение, разработанное специально для использования в ветеринарных клинических лабораториях, предоставляет множество инструментов для повышения эффективности и точности. Эти программы могут предлагать отслеживание образцов, управление запасами, управление домашними животными и инструменты для соблюдения нормативных требований. Эти программы также могут улучшить коммуникацию, управление
Назначение и доступ пациента к результатам теста с большей легкостью и скоростью. Короче говоря, использование программного обеспечения в ветеринарной клинике повышает эффективность, точность, контроль и удобство работы.
пациенты.

\subsection{Преимущества веб-сайта в ветеринарная клиника}

Иметь веб-страницу означает быть в Интернете, а это, в свою очередь, означает быть везде, поскольку сеть разрушает любые пространственные барьеры. Имея сайт вашей ветеринарной клиники, вас может найти кто угодно, независимо от того, где он физически находится.

Использование программного обеспечения в клинической лаборатории может обеспечить ряд преимуществ, некоторые из которых включают:

\begin{itemize}
\item Более высокая эффективность: программное обеспечение позволяет автоматизировать многие рутинные задачи, уменьшить количество ошибок и уменьшить потребность в контроле;
\item Более эффективное управление данными: Ветеринарные клинические лаборатории
обрабатывать и хранить большие объемы данных о домашних животных, включая результаты анализов, системы могут помочь более эффективно собирать и хранить эти данные,
облегчение выявления закономерностей и тенденций
\item Улучшить качество работы: использование программного обеспечения позволяет уменьшить количество ошибок и повысить объективность результатов;
\item Соответствие нормативным требованиям: Ветеринарные клинические лаборатории
подчиняются многочисленным правилам и стандартам, и использование специализированного программного обеспечения может помочь обеспечить соблюдение этих правил.
\end{itemize}


В настоящее время технологии развиваются день ото дня, при этом наблюдаются изменения там, где есть важность разработки новых технологий для управления и наблюдения за тем, как компании и клиники должны выполнять процессы, поэтому для достижения поставленных целей необходимо использовать технологические инструменты.

Преимущества, которые технологические инструменты дают ветеринарной клинике:

\begin{itemize}
\item  Предотвратите трату времени персоналом клиники на поиск информации, так как при ручном поиске может возникнуть риск предоставления неверных данных.

\item  Существует значительный интерес к переходу на новые рабочие процессы из-за простоты и эффективности, которые веб-приложения обеспечивают для получения и обработки информации.
\end{itemize}

\subsection{Исследование предметной области}
На этапе анализа наиболее важные требования расставляются по приоритетам и определяются, чтобы получить приблизительное представление о проекте и объеме, который будет иметь приложение.

Для этого в ветклинике, в которой были получены основные требования центра, что позволило определить подходящую модель для создания технологического решения согласно существующим потребностям в ветклинике.

Где можно было узнать проблему, влияющую на упорядочение процессов, которыми занимаются в ветклинике. В этом случае именно ветеринар ведет учет продажи продукции, консультаций, лечения, прививок, дегельминтизации, стрижки собак и операций; все эти данные регистрировались вручную.

Затем, получив информацию от профессионального врача, была разработана база знаний с информацией об элементах, которые будут использоваться для создания программного обеспечения. А также необходимые ресурсы для создания веб-инструмента, например; язык программирования, тип базы данных, инструменты проектирования, прототипы, библиотеки и API.
