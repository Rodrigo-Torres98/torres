\newpage
\begin{center}ВВЕДЕНИЕ\end{center}

\addcontentsline{toc}{section}{ВВЕДЕНИЕ}

Ветеринарные клиники продолжают использовать бумажные документы для хранения информации о Однако у больных животных этот метод может привести к ошибкам записи и затруднениям при передаче и считывании информации. Пациентам и ветеринарам приходится обращаться в ветеринарную клинику или физическую лабораторию для получения результатов, что может задержать лечение пациента.
для решений данной проблемы было проведено исследование, в ходе которого было предложено решение, облегчающее управление процессами и процедурами с
используя современные технологии, например, создание приложения
хранить и редактировать информацию, введенную врачами.

Как и другим медицинским учреждениям, ветеринарным клиникам необходимо программное обеспечение, помогающее им управлять административными функциями и информацией о пациентах. От первоначального контакта с владельцами домашних животных до напоминаний, ветеринарное программное обеспечение доступно для оптимизации операций и облегчения работы с пациентами.

Предлагаемое решение ускорит сроки и снизит процент типографские ошибки и потеря физических документов. Для достижения этой цели будет поощряться использование компьютера и подключения к Интернету. В работу будет входить разработка и реализация алгоритма, связанного с формированием и приемом информации. Кроме того, будет разработан алгоритм поиска для каждого отделения клиники, в том числе для ветеринарных врачей больниц.

В общем, цель этой статьи — предложить решение проблемы поддержания
медицинские записи, используя современные технологии для облегчения процессов и процедур, а также уменьшения ошибок и потери информации.
Целью данной работы является разработка сайта для госпиталя и ветклиники компании «Veterinaria Los Héroes», облегчение и ускорение оформления информации. Для достижения этой цели необходимо решить
следующие задачи:

\begin{itemize}
	\item анализировать тематическое направление;
	\item разработать структуру сайта;
	\item внедрить структуру, разработанную с помощью веб-технологий с\linebreak
	с помощью CMS.
\end{itemize}

Структура и объем работы. Доклад состоит из введения, 4 разделов основной части, заключения, списка использованных источников. Текст выпускной квалификационной работы состоит из 55 страниц, из 44 страницы основного текста.
Первый раздел на этапе описания технической характеристики тематического направления содержит обобщение информации о деятельности клинической лаборатории ветлечебницы по ее развитию.

Второй раздел, на этапе технического задания, содержит требования к развивающийся сайт.
Третий раздел, на стадии технического проектирования, содержит дизайнерские решения сайта.

В четвертом разделе приводится перечень классов и их методов, используемых при разработке сайта, проводится тестирование разработанного сайта.
В заключении основные результаты работы получены
во время разработки.
